% This is LLNCS.DEM the demonstration file of
% the LaTeX macro package from Springer-Verlag
% for Lecture Notes in Computer Science,
% version 2.4 for LaTeX2e as of 16. April 2010
%
\documentclass{llncs}
%
\usepackage{makeidx}  % allows for indexgeneration
%
\begin{document}
%
\frontmatter          % for the preliminaries
%
\pagestyle{headings}  % switches on printing of running heads

\mainmatter              % start of the contributions
%
\title{Untangling email conversations}
%
\titlerunning{Hamiltonian Mechanics}  % abbreviated title (for running head)
%                                     also used for the TOC unless
%                                     \toctitle is used
%
\author{Tim Repke \and Ralf Krestel}
%
\authorrunning{Tim Repke et al.} % abbreviated author list (for running head)
%
%%%% list of authors for the TOC (use if author list has to be modified)
\tocauthor{Tim Repke, Ralf Krestel}
%
\institute{Hasso Plattner Institute, Potsdam, Germany\\
\email{(tim.repke|ralf.krestel)@hpi.de}}

\maketitle              % typeset the title of the contribution

\begin{abstract}
The abstract should summarize the contents of the paper
using at least 70 and at most 150 words. It will be set in 9-point
font size and be inset 1.0 cm from the right and left margins.
There will be two blank lines before and after the Abstract. \dots
\keywords{computational geometry, graph theory, Hamilton cycles}
\end{abstract}
%
\section{Fixed-Period Problems: The Sublinear Case}
%
citing something \cite{headerless,signature,similarity,zones}
%
\subsubsection{The General Case: Nontriviality.}
ads

\bibliographystyle{splncs03}
\bibliography{biblio} 
\end{document}
