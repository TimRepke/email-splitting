% This is LLNCS.DEM the demonstration file of
% the LaTeX macro package from Springer-Verlag
% for Lecture Notes in Computer Science,
% version 2.4 for LaTeX2e as of 16. April 2010
%
\documentclass{llncs}
%
\usepackage{makeidx}  % allows for indexgeneration
\usepackage{fancyref}
\usepackage{graphicx}
\usepackage{subcaption}
\captionsetup{compatibility=false}
%

\newcommand{\dummyfig}[3]{
	\centering
	\fbox{
		\begin{minipage}[c][#1\textheight][c]{#2\textwidth}
			\centering{#3}
		\end{minipage}
	}
}
\begin{document}
%
\frontmatter          % for the preliminaries
%
\pagestyle{headings}  % switches on printing of running heads

\mainmatter              % start of the contributions
%
\title{Untangling Email Conversations}
%
\titlerunning{Untangling Email Conversations}  % abbreviated title (for running head)
%                                     also used for the TOC unless
%                                     \toctitle is used
%
\author{Tim Repke \and Ralf Krestel}
%
\authorrunning{Tim Repke et al.} % abbreviated author list (for running head)
%
%%%% list of authors for the TOC (use if author list has to be modified)
\tocauthor{Tim Repke, Ralf Krestel}
%
\institute{Hasso Plattner Institute, Potsdam, Germany\\
\email{(tim.repke|ralf.krestel)@hpi.de}}

\maketitle              % typeset the title of the contribution

\begin{abstract}
The abstract should summarize the contents of the paper
using at least 70 and at most 150 words. It will be set in 9-point
font size and be inset 1.0 cm from the right and left margins.
There will be two blank lines before and after the Abstract. \dots
%\keywords{computational geometry, graph theory, Hamilton cycles}
\end{abstract}
%
\section{Introduction}
früher: emails strukturiert, eingerückt, etc; heute: sehr frei/divers, geht nicht mehr mit regeln; Ansatz: deep learning

ALTERNATIV:

emails used for SNA, classification, behaviour, profiling, and more

most based on full mails, but conversations contain quoted messaged including additional metadata; distracts downstream tasks

our goal: clean up emails properly, don't assume full corpus, therefore keep all information

\subsection{Problem statement}
describe task: split mails in different parts of the thread; parse meta information in intermediate headers

\section{Related Work}
\cite{rfa} (detecting request for action) works better with zoning (see \cite{zones}), removes noise and focuses on text itself (excluding signature etc)

\cite{zones} "Zebra" detects nine different zones, only last email in thread is considered, rest is quoted, classifies lines individually with features using SVM, three zones with 0.91 accuracy, 9 zones 0.87 accuracy; context features didn't add improve performance

\cite{profiling} don't describe much, uses CRF inspired by \cite{signature}, five categories (author text, signature, advertisement, quoted text, reply lines), for eval using three zones with accuracy 0.88, F1 0.9, compared with \cite{signature} 0.64 and 0.75 on their data

\cite{signature} "Jangada" detects signature and reply lines, only considers the last $n$ lines, using CRF with 0.97 accuracy, CPerceptron with 0.989 accuracy; extraction task only performed on mails known to contain signatures! used 20 newsgroup dataset \cite{20news}

\cite{headerless} look for overlaps in enron dataset to split mails into parts of the conversation; closest to our task; we don't consider full dataset though (looking at each mail individually); their goal is to then reassemble the conversation threads, tested on 20 threads (leading to 465 mails), claim accuracy approaching 100\%

\cite{similarity} similar to \cite{headerless}

[citation needed] previous social network analysis wrong, because not considering full picture

\cite{workhard} overview of recent technologies applied to enron and avocado (personal business classification + SNA)

\cite{replying} also recent, on avocado, reply behaviour, only "main header"

\cite{enron} enron corpus

\cite{avocado} avocado corpus

[citation needed] shneiderman corpus

\cite{20news} 20 newsgroups

\section{Segmentation of Emails}
describe an overview

\subsection{Representation of Email Data}
learn char or line embedding

\subsection{Classification of Email Lines}
continue training for classification into header/body

how to split thread based on that

\subsection{Extraction of Email Meta-Data}
continue training for further classification of body and header parts individually

\section{Experimental Setup}

\subsection{Dataset}
describe how data is selected from enron corpus (800: 500 train, 200 test, 100 eval)

annotated very detailed, even with some entity linking and signature parts, considered level of detail: header, body, signature, intro/outro; one annotator

some numbers comparing raw data and contained information in \Fref{tab:dataset}

\begin{table}
	\centering
	\caption{Experimental Results}
	\label{tab:dataset}
	\begin{tabular}{|c|ccc|}
		\hline
		Approach & Train & Test & Eval\\ \hline
		Raw Mails& 500 & 200 & 100\\
		Actual num mails & ? & ? & ? \\
		Avg Threadlen & ? & ? & ?\\
		signatures & ? & ? & ? \\
		people extracted (raw) & ?(?) & ?(?) & ?(?)\\
		\hline	
	\end{tabular}
\end{table}

\subsection{Competing Approaches}
baseline (regex+rules)

reimplement zebra\footnote{\url{http://zebra.thoughtlets.org/zoning.php}} and jangada\footnote{\url{http://www.cs.cmu.edu/~vitor/software/jangada/}}

describe differences, what needs to be changed to make it fair

RNN with features

show which task is more or less comparable between all

\subsection{Learning Model Parameters}

\begin{figure}
	\centering
	\begin{subfigure}[t]{0.5\textwidth}
		\centering
		\dummyfig{0.17}{0.9}{xaxis=perturbation, yaxis=accuracy, one line/boxplot} 
		\caption{Obfuscating test samples}
	\end{subfigure}%
	~ 
	\begin{subfigure}[t]{0.5\textwidth}
		\centering
		\dummyfig{0.17}{0.9}{xaxis=perturbation, yaxis=accuracy, three lines for different perturbation during eval} 
		\caption{Training with perturbation}
	\end{subfigure}
	\caption{Tolerance of the trained model against perturbation of the input}
\end{figure}

size of model/dimensions

size of training dataset (how much annotated data needed)

cross corpus (?)

\section{Results}
evaluate different complexities: full task (get all information), split head+body, additionally split body parts (sig, intro, outro)

results in \Fref{tab:results-comp} and \Fref{tab:results}

\begin{table}
	\centering
	\caption{Results of Two-Zone Task}
	\label{tab:results-comp}
	\begin{tabular}{|c|cc|}
		\hline
		Approach & Dataset& result\\ \hline
		We 1& a & 0.98\\
		We 2& a & 0.98\\
		We 3& a & 0.98\\
		Zebra\cite{zones} & d & 0.78\\
		Jangada\cite{signature} & d & 0.8\\
		\hline	
	\end{tabular}
\end{table}

\begin{table}
	\centering
	\caption{Experimental Results}
	\label{tab:results}
	\begin{tabular}{|c|cc|}
		\hline
		task & prec/rec& f1\\ \hline
		2zone& a & 0.98\\
		5zone& a & 0.98\\
		full& a & 0.98\\
		\hline	
	\end{tabular}
\end{table}

\Fref{fig:mpd} shows influence of extracting information vs using raw data

\begin{figure}
	\centering
	\begin{subfigure}[t]{0.5\textwidth}
		\centering
		\dummyfig{0.17}{0.9}{xaxis=timebuckets, yaxis=freq} 
		\caption{raw data}
	\end{subfigure}%
	~ 
	\begin{subfigure}[t]{0.5\textwidth}
		\centering
		\dummyfig{0.17}{0.9}{xaxis=timebuckets, yaxis=freq} 
		\caption{extracted data}
	\end{subfigure}
	\caption{Mail Frequency}
	\label{fig:mpd}
\end{figure}


maybe: email classification with+without splitting (i.e. try to classify folders)

\section{Conclusion and Future Work}
nicht nur für mails, auch forenthreads und andere semi-strukturierte daten



\bibliographystyle{splncs03}
\bibliography{biblio} 
\end{document}
